%% Conclusion
\section{Conclusion}

Traditional networks are complex and hard to manage.  One
of the reasons is that the control and data planes are vertically
integrated and vendor specific. Another, concurring reason, is that
typical networking devices are also tightly tied to line products and
versions.  In other words, each line of product may have its own
particular configuration and management interfaces, implying long
cycles for producing product updates (e.g., new firmware) or upgrades
(e.g., new versions of the devices). All this has given rise to vendor
lock-in problems for network infrastructure owners, as well as posing
severe restrictions to change and innovation.

Software-Defined Networking (SDN) created an opportunity for solving
these long-standing problems.  Some of the key ideas of SDN are the
introduction of dynamic programmability in forwarding devices through
open southbound interfaces, the decoupling of the control and data
plane, and the global view of the network by logical centralization of
the ``network brain''.  While data plane elements became dumb, but
highly efficient and programmable packet forwarding devices, the
control plane elements are now represented by a single entity, the
controller or network operating system.  Applications implementing the
network logic run on top of the controller and are much easier to
develop and deploy when compared to traditional networks. Given the
global view, consistency of policies is straightforward to enforce.
SDN represents a major paradigm shift in the development and
evolution of networks, introducing a new pace of innovation in networking infrastructure.

In spite of recent and interesting attempts to survey this new chapter
in the history of networks~\cite{lara2014,jarraya2014,nunes2014}, the
literature was still lacking, to the best of our knowledge, a single
extensive and comprehensive overview of the building blocks, concepts,
and challenges of SDNs.
%
%% , such as network operating systems, network hypervisors,
%% programming languages, scalability and security challenges,
%% debugging and troubleshooting, challenges and open roads.
%
Trying to address this gap, the present paper used a layered approach
to methodically dissect the state of the art in terms of concepts,
ideas and components of software-defined networking, covering a broad
range of existing solutions, as well as future directions.

We started by comparing this new paradigm with traditional networks
and discussing how academy and industry helped shape software-defined
networking.  Following a bottom-up approach, we provided an in-depth
overview of what we consider the eight fundamental facets of the SDN
problem: 1) hardware infrastructure, 2) southbound interfaces, 3)
network virtualization (hypervisor layer between the forwarding
devices and the network operating systems), 4) network operating
systems (SDN controllers and control platforms), 5) northbound
interfaces (common programming abstractions offered to network
applications), 6) virtualization using slicing techniques provided by
special purpose libraries and/or programming languages and compilers,
7) network programming languages, and finally, 8) network
applications.

%%PJV:  please someone clarify content of (6) w.r.t. ``virtualization'' in (3) but is a more concise way

SDN has successfully managed to pave the way towards a next generation networking, spawning an innovative research and development
environment, promoting advances in several areas: switch and
controller platform design, evolution of scalability and performance
of devices and architectures, promotion of security and dependability.

We will continue to witness extensive activity around SDN in the near
future. Emerging topics requiring further research are, for example:
the migration path to SDN, extending SDN towards carrier transport networks, realization of the
network-as-a-service cloud computing paradigm, or software-defined
environments (SDE).
\coloredtext{As such, we would like to receive feedback from the networking/SDN community as this novel paradigm evolves, to make this a ``live document'' that gets updated and improved based on the community feedback.
We have set up a github page\footnote{\coloredtext{https://github.com/SDN-Survey/latex/wiki}} for this purpose, and we invite our readers to join us in this communal effort.}

\begin{comment}

Traditional networks have always been complex and hard to manage.  One
of the reasons is that the control and data planes are vertically
integrated and vendor specific, potentializing vendor lock-in problems
for network infrastructure owners and posing severe restrictions to
change and innovation.  Another reason is that typical networking
devices are also tightly tied to line products and versions.  In other
words, each line of product may have its own particular configuration
and management interfaces.  Notwithstanding, innovation and
flexibility are hard to achieve due to long cycles for producing
product updates (e.g., new firmware) and upgrades (e.g., new versions
of the devices), which are many times dependent on hardware
modifications.

Software-Defined Networking (SDN) was born as a new way for solving
long standing problems, such as lack of flexibility and unacceptable
innovation cycles.  Some of the key ideas of SDN are the introduction
of dynamic programmability in forwarding devices through open
southbound interfaces, the decoupling of the control and data plane,
and the global view of the network by a logical centralization of the
``network brain''.  While data plane elements became dummy, highly
efficient and programmable packet forwarding devices, the control
plane elements is now represented by a simple entity, the controller
or network operating system.  Applications implementing the network
logic run on top of the controller and are much easier to develop and
deploy when compared to traditional network.  This flexibility
introduced a new world of possibilities and a new pace of innovation
in networking infrastructures, i.e., a major paradigm shift on the
development and evolution of networks.

In spite of different attempts to depict the big picture and building
up a thorough survey of this new chapter on the history of
networks~\cite{lara2014,jarraya2014,nunes2014}, there is yet no single
extensive and comprehensive overview of the building blocks, concepts
and challenges of SDNs, such as network operating systems, network
hypervisors, programming languages, scalability and security
challenges, debugging and troubleshooting, challenges and open roads.
To try to cover SDN in-deep, we used a layered approach to extensively
and comprehensively dissect the state of the art in terms of concepts,
ideas and elements of software-defined networking, covering a broad
range of existing solutions, challenges and future directions.

By rendering the roots of SDN and its history, we started by comparing
this new paradigm with traditional networks and introducing a
broadening industry-driven definition of software-defined networking.
Following a bottom up approach, we provided an in-deep overview of
what we considered the eight part of the SDN problem: 1) hardware
infrastructure, 2) southbound interfaces, 3) network virtualization
(hypervisor layer between the forwarding devices and the network
operating systems), 4) network operating systems (SDN controllers and
control plat- forms), 5) northbound interfaces (to offer a common
program- ming abstraction to the upper layers, mainly the network
applications), 6) virtualization using slicing techniques provided by
special purpose libraries and/or programming languages and compilers,
7) network programming languages, and finally 8) management
applications.

The two first layers introduce the forwarding devices and southbound
interfaces, which represent the early driving forces of this paradigm
shift in networking.  OpenFlow is clearly the most widely deployed and
accepted southbound interfaces, with several dizen of commercial
product and free software available in the market.  Despite of that,
it is not the only southbound API.  Other examples include POF,
ForCES, and OpFlex.

Going up on the layered approach, network hypervisors and network
operating systems are thoroughly discussed.  This is certainly one of
our major contributions compared to previous attempts of introducing
SDN.  Both network hypervisor and network operating systems are
fundamental and critical pieces of the SDN puzzle.  Therefore, we have
carefully dissected their essential properties, system design
requirements and challenges.

Network operating systems are connected to forwarding devices through
the southbound API and provide a northbound API for network
applications.  Beyond that, they provide essential services for higher
level control applications, such as shortest path forwarding, topology
manager, statistics manager, device manager, notification manager, and
security mechanisms.  It is also worth emphasizing that controllers
provide a global view of the network topology and make it easy for
network applications to control the network infrastructure.

As northbound APIs we have programming languages or other simpler interfaces such as REST APIs.
%The main advantage of programming languages is their capability of abstracting low level instruction set' (e.g., OpenFlow) to simplify the development of network applications.
%More than that, programming languages are important tools for solving different kinds of problems and challenges, such as access control rules, security policies, fault tolerance, consistent updates, sequential and parallel application composition, decoupled multi-tasks for correct network operation, long-term software evolution and maintenance, application state management.
High level network programming languages can be utilized to simplify
the programming of forwarding devices by creating a proper higher
level of abstraction. These languages can help programmers to become
more productive and focus on the problem solving rather than low level
networking details. Initiation of network application developers,
modular software engineering, code reuse, and new opportunities for
research and industry are key benefits of network programming
languages. Most of the existing proposed programming languages focus
on OpenFlow. The predominant programming paradigm is the declarative,
with few exceptions (e.g. Pyretic) based on interpretive
languages. Traffic engineering, mobility and wireless networks,
measurement and monitoring, security and dependability, and data
center networking are representative types of management applications
were deeply treated.

Without exception, debugging and troubleshooting has always been an
important topic in computing infrastructures, parallel and distributed
systems, and computer networks.  However, networks were still in the
``stone age'' before SDN.  Hopefully, this as been changing rapidly
with the several SDN/OpenFlow driven debugging, troubleshooting,
testing and verification suites, and simulation and emulation
platforms.  These tools have a significant impact on the development
and evolution of networks since developers are now armed with
resources and techniques similar to traditional computing systems,
while on the past (tradicional networks) it was really challenging to
debug, verity or have emulation platforms to enable the off-line
development of new protocols and applications that can be easily
deployed on a production network without requirement any modification.

In addition to many outcomes that we discussed in our layered
approach, SDN has successfully managed to pave the way towards an
innovative research and development environment.  Switch designs,
controller platforms, scalability of SDN, performance evaluation of
devices and architectures, security and dependability, organizational
and migration barriers, and finally extending SDN towards carrier
transport networks, cloud computing providers and software-defined
environments (SDEs) are some of the areas that need more attention
with respect to further research and investigations.  In particular,
software-defined environments are emerging as one key strategy for
changing the way IT infrastructures and build and operated.  SDEs
focus on the business requirements and needs rather than the business
being restricted due to a diversity of constraints (e.g., static
nature of network, lack of knowledge of IT operators, excessive or
unnecessary complexity, restrictive IT grounded rules) such as those
created by technology or IT teams (e.g., network operation, system
administration, development, security) with no common interface to
promote interoperability and innovative solutions required by the
enterprise.

\end{comment}

%In addition, we also look at cross-layer problems such as debugging and troubleshooting mechanisms. The discussion in Section V on ongoing research efforts, challenges, future work and opportunities concludes this paper..



%SDN has emerged as an approach for doing network experimentation in campus Networks. Once its potential was explored, SDN has grown into a new way of thinking about and  deploying networks. SDN has revolutionized networking through two main contributions. The first one is the concept of a network operating system (NOS), that will make the life of application developers much easier by providing network support, allowing innovation to happen not only at the edge, but within the network. The second contribution of SDN is the development of open and standardized APIs, truly opening the network to the application logic. As an emerging area in networking, much ongoing work and challenges are still to be faced by SDN. As shown in this survey, SDN is reshaping the networking  world and opening a myriad of new opportunities, both from a technical as well as from a business perspective.

%\newpage

%\section{MISC}
%
%\subsection{Taming your network infrastructure with SDN}
%
%\subsection{A clear and separated control plane}
%
%One of the keys for the SDN success in the clear separation between control and data plane.
%Further, the abstraction of a centralized controller is fundamental to simplify network development and operation.
%It is much easier to write applications without having to care about distributed data gathering and updates.
%
%Remove the control plane from the devices to software-based servers (controller instances) gives much more freedom, flexibility and can seen as a highway without speed limits anymore.
%Before, the speed was limited to 40 miles per hours due to design and control complexities. 
%Now, with design and control improvements, it looks like a free-style highway, fostering innovation and new services deployment at faster speeds.
%It is now up to the car drivers (i.e. application developers, business, etc.) to decide at which speed they are going to drive their networks and innovations.
%
%The control plane is responsible for the distributed state and specification abstractions.
%Only the forwarding abstraction is left to network devices.
%This removes the complexity from the network equipments to the control software.
%As software is much easier to develop, change and adapt, new network features, protocols, services and architectures can be delivered at software speed, and not anymore at hardware speed, which is very slow and costly.
%
%\subsection{Management applications: smart network brains}
%
%% Implementation and analysis of control and forwarding plane for SDN~\cite{risdianto2012}
%%FIXME
%
%While controllers can be viewed as the network nervous system, management applications represent the network brain.
%A controller only transports control messages between data plane devices and management applications.
%When receiving a new message, an application will analyze its content and choose what to do next.
%It can be an action such as forwarding the message to a second management application or send a response to the device with some configuration information (brain telling what the specific body member should do in reaction to this specific stimulation).
%
%There are already many management applications for SDN as can be seen in section~\ref{sec:whatisoutthere}.
%However, there is still a lack of more advanced and intelligent applications to foster autonomic network management, for instance.
%With SDN, a network can more easily adjust itself in different ways without requiring human intervention.
%Redundancy, recovery actions, aggregated links for augmented throughput, among many other things, are less costly and more flexible to deploy in SDN.
%Thus, it looks to be a good time to push the development of smarter brains for future networks.

%% EOF: 8_conclusion.tex
